\documentclass{article}
\title{Vreetzak gaat naar de ruimte}
\author{Thomas van Maaren}
\date{}
\usepackage[margin=1in]{geometry}

\begin{document}
\maketitle

\section{Leer over het land Kach}

Kijk op het plaatje, daar ligt het hele land Kach. Dit verhaal gaat over maar een paar mensen, over Tijgertje en Knorretje en Winnie de Pooh en over de president. De blauwe olifant Vreetzak.

Kach is een arm land met de minste inwoners van de hele wereld. Het is heel erg heuvelachtig en qua oppervlakte is het het grootste land ter wereld.

\section{Het rare telefoonbericht}

Vreetzak zat achter zijn bureau. Ineens ging één van de drie telefoons. Toen zei Vreetzak: ``Met Vreetzak van Bant". Pooh was aan de telefon en zei: ``Met Pooh. Mijn vriend Tijgertje is weg". Toen zei Vreetzak: ``Dat is nou toevallig. Ik wilde je net bellen en vertellen dat je vriend weg is". Pooh zei: ``Tijgertje zei dat hij naar de ruimte ging, maar hij is niet teruggekomen". ``Oké", zei Vreetzak. ``Ik ga wel naar de ruimte. Alleen, er is één probleem. Ik heb niet genoeg geld". ``Oké", zei Pooh, ``ik geef wel geld aan jou". Pooh had heel veel geld, omdat hij het gouden ei had gevonden. Met dat gouden ei kan je wensen doen. Pooh had een wens om de rijkste beer ter wereld te worden. ``Oké", zei Vreetzak. ``Kom maar hier met het geld. Dan kan ik naar de ruimte". Pooh vond het een goed idee en bracht het geld naar Vreetzak. Vreetzak zei: ``Oké, dan ga ik nu naar www.klussen.kl". De volgende dag zouden de bouwers komen voor het bouwen van de raket.

\section{Het bouwen van de raket}

Vreetzak stapte uit zijn bed en ging naar het regeringsgebouw. Hij zat voor zijn kantoortje en toen kwamen de bouwers naar hem toe. Ze zeiden: ``Waar kunnen we het bouwen?" ``In mijn schuur," zei Vreetzak. De bouwers begonnen met bouwen. Het duurde tien weken. Toen waren ze klaar. Toen kwamen ze naar Vreetzak die in zijn kantoor zat. Ze zeiden: ``Wij zijn klaar".

\section{Vreetzak gaat naar de ruimte}

``Oké", zei Vreetzak. En stapte in de raket. Na een beetje moeite, omdat Vreetzak een beetje zwaar was, steeg de raket op. Na een paar seconden was hij in de ruimte. Hij dacht: Bij welke planeet is Tijgertje eigenlijk?" Hij keek en hij wist welke planeet het meest dichtbij hem was, Mercurius. Hij wist dat zo goed omdat hij fan is van de ruimte. Dus hij vloog naar Mercurius, maar ineens kwam er een ruimte steen op hem af. Hij moest snel weg vliegen. Hij dacht: Nou ja, pech gehad. Dan ga ik maar naar Jupiter. Het is ook een hele grote planeet. Het is de grootste planeet van ons zonnestelsel.

\section{Landing op Jupiter}

Het werd ook heel erg koud, omdat Jupiter verder is van de zon dan onze planeet. Toen ging de telefoon weer. Het was Pooh. Hij vroeg: ``Heb je al een spoor gevonden van Tijgertje?" ``Nee", zei Vreetzak, ``Nog geen spoor". Pooh zei toen: ``Nou jammer". Vreetzak vloog verder. Een paar minuten later landde hij op Jupiter. Hij stapte uit de raken en ging zoeken. Hij zag een rare stad. Hij zag daar ook een raar wezen lopen. Het was een ruimtewezen. Het wezen zei: ``Schoe, schoe". Dat betekent: ``Daar bent u". Hij nam Vreetzak mee naar huis.

\section{Vreetzak gaat in de gevangenis}

Hij nam Vreetzak mee naar een raar gebouw. Het was een politiebureau. Vreetzak wist dat niet omdat alles er anders uitzag. Hij nam hem mee naar de gevangenis en duwde hem erin. Toen zei het ruimtewezen: ``Toe je pi", dat betekent. ``Welkom in de gevangenis". Vreetzak begreep niets van wat die man zei en de man liep weg. Hoe kon Vreetzak er nu nog uit? Ondertussen zag Vreetzak Tijgertje door het raam. Hij riep: ``Tijgertje!". Tijgertje zei: ``Vreetzak, wat doe jij hier?" ``Ik zoek jou", zei Vreetzak. ``En waarom zit jij in de gevangenis?" zei Tijgertje. Toen zei Vreetzak: ``Geen idee". Tijgertje kwam naar boven en deed het haakje open. Toen deed Tijgertje de deur open. Vreetzak stapte uit en zei: ``Dank je wel Tijgertje". ``Graag gedaan", zei Tijgertje. Toen zei Vreetzak: ``Waarom kan je niet meer terug naar Pooh beer?" ``Omdat mijn raket stuk is", zei Tijgertje. ``Kom", zei Tijgertje, ``dan gaan wij nu naar jouw raket". Vreetzak vond het een goed idee en ze liepen poot in poot naar de raket.

\section{Vreetzak is president van Jupiter}

Tijdens het wandelen zagen ze veel mensen. Die juichten en riepen: ``De president is gevangen". Ze zeiden het allemaal in het Jupiters en Vreetzak en Tijgertje begrepen het deze keer wel omdat Vreetzak een beetje Jupiters sprak, omdat hij fan was van de ruimte. Vreetzak dacht even na. Misschien dachten ze dat ik de president was. Het werd tijd om te stemmen voor een andere president. Vreetzak stond ook op de stemlijst, maar iedeereen kee op de stemlijst en ze zagen er een naam bij die ze nooit eerder hadden gezien. Dus iedereen stemde op Vreetzak, omdat hij nieuw was. Dus Vreetzak werd president van Jupiter. Hij zat voor zijn kantoor. Heel vaak ging de telefoon. In Kach ging de telefoon maar heel weinig; maar één keer in de maand. Maar hier ging het bijna elke minuut. Echt een goede president was Vreetzak niet. Eén op de zes mensen was gelukkig. Heel veel mensen waren arm. Vreetzak en Tijgertje waren dus eigenlijk de enige mensen in Jupiter die rijk waren. Een keertje gingen mensen protesteren, ``Weg met de president". Toen zei Vreetzak tegen de Jupiternaren, ``WAAROM?!!!!!!!!!!!" ``Omdat bijna iedereen op de planeet arm is". Toen zei Vreetzak: ``Oké, oké, oké, oké". En hij gaf aan alle mensen die arm waren geld. Toen ging iedereen weg. En Vreetzak was weer oké. Vreetzak vond het eigenlijk niet meer leuk om president te zijn. Hij wilde terug naar Kach. Hij zei tegen de planeet dat hij niet meer president wilden zijn. En ging dus terug naar zijn raket lopen met Tijgertje. En steeg weer op.
\section{Problemen met de zon}

Vreetzak steeg weer op. Toen zei Vreetzak tegen Tijgertje: ``Lekker weer terug naar ons lief planeetje". Per ongeluk ging hij langs zijn planeet in plaat van er naartoe. En het werd heel erg warm in de raket, omdat ze zo dicht bij de zon waren. Vreetzak keek naar buiten en zag zijn eigen planeet niet meer. Toen zei Vreetzak: ``Misschien zijn we langs onze planeet gegaan". Toen ineens werd het zo heet dat de raket een beetje ging smelten. Nu konden ze helemaal niet meer in de raket staan. Toen gingen Vreetzak en Tijgertje uit deraket vallen. En toen ging de mobiel van Vreetzak. Het was Pooh. Hij vroeg: ``Heb jij een spoor gevonden van Tijgertje?". ``Nee", zei Vreetzak, ``wij hebben geen spoor gevonden, maar wel Tijgertje zelf". ``Wow, echt waar!!!!!!", zei Pooh. ``Ik dacht dat ik Tijgertje nooit meer zou zien". En Pooh huilde van geluk. ``Nou, breng Tijgertje dan bij mij". ``Oké, maar is nog een probleempje", zie Vreetzak. ``Onze raket is gesmolten". ``Oké", zei Pooh, ``Ik ga onmiddelijk naar jullie toe. Eén momentje. Waar zijn jullie?" ``Dichtbij dezon", zeiden Vreetzak en Tijgertje tegelijk. ``Oké, dan kom ik eraan", zei Pooh. Na een tijdje kwam Pooh er aan met zijn raket. ``Kom binnen", zei Pooh. ``Oh Tijgertje, ik ben zo blij om jou te zien. Kom mee. Kom lekker naar onze eigen planeet". Na een tijdje werd het steeds kouder. En u was het wel erg koud. ``Eindelijk", zei Pooh, ``daar is onze planeet".

\section{Landing op Pluto}

Toen ze uit Pooh's raket kwamen leek het echt niet op de aarde. Toen dacht Vreetzak even na. ``Dit is bruin en Saturnus is bruin dus dat betekent dat wij op Saturnus zijn". Toen zei Pooh meteen: ``Maar dit is veel te koud om Saturnus te zijn, dit is Pluto". ``Oh ja", zei Vreetzak, ``dat had ik zelf kunnen bedenken. Iets is weer uit de lucht komen vallen".

Na één kilometer lopen kwamen ze bij een dorpje. ``Oh nee", zei Vreetzak, ``ons eten is op". Wat moeten we doen zonder eten?" zei Pooh. ``Geen idee", zei Tijgertje. Toen dacht Vreetzak even na. ``Ik weet het niet," zei Vreetzak. Pooh dacht even na, ``Umm..... De tovenaar's hoed stelen". Tijgertje en Vreetzak vonden het een goed idee en gingen naar het huis van de tovenaar.

Gelukkig was de deur open en ze kropen stiekem naar binnnen. Vreetzak pakte de hoed van de tovenaar en Tijgertje pakte het boek vvan de tovenaar en ook de toverstaf van de tovenaar. Ineens kwam de tovenaar naar binnen. Hij dacht: ``Waar is mijn toverstaf en waar is mijn boek en waar is mijn hoed?" Hij hoorde een klein gegiechel. Hij keek onder de tafel. Hij vond Tijgertje. ``Tjon tje!!!!!" Dat betekent: ``Waarom heb jij mijn boek?" Hij pakte zijn knuppel en gaf bijna een klap op tijgertje. Tijgertje ging net op tijd weg.

\section{Vreetzak en Knorretje}

Toen ineens pakte de tovenaar het boek en de hoed en de toverstaf en schopte die allemal naar buiten.

``Nou", zei Pooh, ``nu weet ik het niet meer". Toen was het weer tijd dat Vreetzak ging nadenken. ``Wacht eens even. Als wij nou eens Knorretje bellen en vragen of hij hier komt en nog eten meeneemt dan komt het wel goed. Dus even nog naar de raket lopen en mijn mobiele telefoon pakken". ``Als Knorretje maar snel komt, want ik heb zo'n honger".

Eindelijk kwamen ze bij de raket. Maar de raket was helemaal ingestort. ``Dat heeft zeker de tovenaar gedaan", zei Vreetzak. ``Maar dat maakt niks uit," zei Pooh, ``want Knorretje komt toch ook?" ``Ja, je hebt gelijk," zei Vreetzak. ``Alleen nu even de mobiele telefoon zoeken, tussen al dat puin". Na een uur zoeken hadden ze uiteindelijk de mobiele telefoon gevonden. Maar toen hadden ze echt honger. En dus belde Vreetzak Knorretje. ``Hij komt eraan", zei Vreetzak. ``Gelukkig", zei Tijgertje, ``want ik heb zo'n dorst en honger".

Knorretje was heel erg bang toen hij in de raket stapte. Na twee uur kwam hij aan op Pluto. ``Aaah... eten", zei Tijgertje. En ze vonden het eten heel erg lekker. En Vreetzak was iets dikker dan ervoor.

``Nou zullen we weer opstijgen?" zei Vreetzak. ``Oké", zei iedereen tegelijker tijd. En iedereen stapte in de raket van Knorretje.

\section{Landing op de Maan}

``Oh Tijgertje", zei Knorretje, toen ze in de raket waren. ``Ik heb je zo gemist". Na een tijdje waren ze aan het kiezen of ze op de maan wilden landen of op de aarde. ``Het zou eigenlijk leuk zijn om op de maan te landen", zei Vreetzak. Dus ze gingen naar de maan.

Na een tijdje gingen ze lunchen en aan het einde was Vreetzak nog dikker. ``Ik hoop dat je nog door de deur naar buiten kan", zei Tijgertje. ``Oh, ik hoop het wel", zei Vreetzak en hij lachte. Gelukkig kon Vreetzak nog wel naar buiten. Dus ze maakten een wandeling op de maan. Na een tijdje zagen ze een tunnel. Vreetzak zei: ``Het zou zo avontuurlijk zijn om daar in te gaan". Dus ze gingen erin. Uiteindelijk zakte Vreetzak in de grond, omdat hij zo zwaar was. ``HELP, HELP, HELP!!!!!!!!" zei Vreetzak. ``Gelukkig heb ik een touw mee", zei Pooh. ``Ik wist wel dat dit soort dingen zouden kunnen gebeuren. Ik doe het in het gat waar Vreetzak doorheen ging. Vreetzak hoor je mij....hoor je mij....hoor je mij?" ``Ja....ja....ja....ja" ``Oké en pak het touw Vreetzak". ``Maar ik zie niks", zei Vreetzak. ``Jij hebt toch een zaklamp, gebruik de zaklamp", zei Pooh. ``Oh ja", zei Vreetzak, en hij pakte de zaklamp. En Vreetzak kroop naar het touw. Pooh trok hem naar boven. Hij is ook de sterkste beer ter wereld. ``Zo", zei Vreetzak, ``Eindelijk weer veilig. We kunnen maar beter weer naar de raket, voordat ons nog iets overkomt. En dus liepen ze weer naar de raket.

\section{Een boef in de raket}

En dus, gingen ze weer in de raket en startten weer op, op weg naar de aarde. ``Drie, twee, één, GO!", zei Vreetzak. Toen steeg de raket weer op.

Ondertussen kwam er iemand van onder een bed vandaan. En tikte op Knorretje's schouder. Hij zei: ``Ga terug naar de tovenaar op Pluto". ``Welke tovenaar?", vroeg Knorretje met een zacht stemmetje. (Want je weet wel Knorretje is niet bij de tovenaar op Pluto geweest). ``Vertel het nou", zei de boef die onder het bed vandaan kwam. ``Ik weet het niet, dus ik vraag het even aan Pooh". Toen kwam Pooh eraan, ``Wat is er precies aan de poot?", vroeg Pooh. ``Aan de poot, je bedoelt zeker aan de hand", zei de boef. ``Oh ja, jij hebt natuurlijk een poot en geen hand. Oké, jijweet wel hé, jullie hebben de spullen van de tovenaar gestolen en de tovenaar wil dat jullie even bij hem komen om te praten: ``Dat doe ik niet", en schopte de boef naar buiten.

\section{Landing op de aarde}

Na een tijdje was het weer tijd om te eten. EnVreetzak werd nog dikker. Het was weer tijd om te slapen. Ze hoorden: ``Boem, boem, boem". Van al dat geluid werd Vreetzak wakker en stapte een beetje slaperig uit zijn bed en keek door de raam. Hij zag de raket van de tovenaar. Hij probeerde te schieten op de raket van Knorretje. Hij maakte Knorretje, Tijgertje en Pooh ook wakker. Vreetzak ging achter het stuur zitten en Pooh en Tijgertje en Knorretje keken door het raam. En dus maakten ze een looping. En toen schoten ze weer en ging de raket van de tovenaar stuk. Maar hij had nog magische krachten. Hij pakte zijn hond en zei dat hij een super blaf moest laten horen. En toen vielen ze allemaal op de aardbol. Gelukkig was het kermis in Kach. En toen vielen ze op het springkussen op de kermis en dat heeft hun levens gered want anders zouden ze dood zijn, omdat de aarde zo hard is.

\section{Het grote feest}

``Hoera, Vreetzak is terug en Tijgertje ook!!!!!!!!!!!!!!!!!!!", riep de hele stad in koor. En Knorretje, Pooh en Tijgertje en Vreetzak kwamen allemaal op het grote podium. En Vreetzak vertelde het hele verhaal wat ze mee hadden gemaakt.

``Je moet een boek gaan schrijven, het is hartstikke spannend hoor", zei een kindje, één van de kinderen van Mama Tjotja.
\end{document}
